\chapter{DePINC's Technical Solution}
\subsection{Algorithms and mechanism}
\begin{flushleft}
    DePINC believes that low power consumption can also provide sufficient trust for algorithms, ensuring widespread use of cryptocurrencies in more future scenarios. By integrating multiple efficient protocols, DePINC continues to drive the development of decentralized technology, offering users a better experience and broader application prospects.
\end{flushleft}
\begin{flushleft}
    DePINC's Algorithmic Consensus Mechanism The core proof algorithm of DePINC's consensus protocol is PoST, which utilizes Plot files for mining. This approach supports DePINC and enhances network miners' income through composite security with other similar algorithmic projects.
\end{flushleft}
\subsection{The network parameters}
\begin{flushleft}
    Conditioned Proof of Space Time, or CPoST, would lead the miners, mining pools, the foundation and other participants to engage in a positive business cycle, so that the whole system would always have a dominant temporary commercial vested beneficiary (this vested beneficiary could change with variables such as time, price and mining difficulty) to promote the whole ecosystem.
\end{flushleft}
\begin{tabular}{ p{4cm} p{8cm} }
    \hline
    \textbf{Total supply}                               & 84 million                                                        \\[5pt]
    \rowcolor{lightgray!30}\textbf{Development team}    & 2.1 million. Way: pre-mined                                       \\[5pt]
    \rowcolor{lightgray!30}\textbf{Miner}               & 80 million. Way: mining                                           \\[5pt]
    \textbf{Avg. block time}                            & 3-4 minutes                                                       \\[5pt]
    \rowcolor{lightgray!30}\textbf{Initial block size}  & 15 DePINC / Block, 2MB block size                                 \\[5pt]
    \textbf{Halve period}                               & In 4 years, the first halving time is about 582688 block height   \\[5pt]
    \rowcolor{lightgray!30}\textbf{Current TPS}         & 70 transactions / sec                                             \\[5pt]
    \textbf{Stake}                                      & Stake amount relate to the computing power                        \\[5pt]
    \rowcolor{lightgray!30}\textbf{No stake}            & Only $10\%$ of the block reward.                                  \\[5pt]
    \hline
\end{tabular}
\section{DePINC Economic Model}
\begin{flushleft}
    DePINC's economic model / consensus mechanism has been upgraded based on the Chia PoST (Proof of Space Time), and is called: CPoST (Conditioned-Proof of Space Time).
\end{flushleft}
\begin{flushleft}
    The model will solve the problems listed below:
\end{flushleft}
\subsection{Economic Model Attack}
\begin{flushleft}
    The main purpose of miners mining is the payback period, and the benefits will inevitably lead to the sale of all mining output, resulting in market crash, lower prices and thinner profits. The CPoST mining model binds miner to its ecosystem, and uses output of mining as future input of mining, to make the entire DePINC system grow automatically.
\end{flushleft}
\subsection{PoW High Maintenance Cost}
\begin{flushleft}
    It requires a huge amount of power to keep the chain with PoW consensus safe. In good days, it works fine for each part of the system, but in hard times, miners have to pay bills by selling, and it is not easy to keep the miners in the system, if they always have to consider how much energy has been consumed.
\end{flushleft}
\subsection{Lack of Long-Term Economic Incentive}
\begin{flushleft}
    Without operational incentive funds, the promotional efficiency and market confidence is low. even the core technology might fail to get continuous update. As a result, effective development and iterations are non-existent in the long-run, the team may even create a fork in the subsequent version, and users will no longer be able to tell which is the main-net.
\end{flushleft}
\subsection{Mining Machine Monopoly}
\begin{flushleft}
    The PoW consensus mechanism will inevitably lead to a race for mining machines. In order to obtain higher hash power, special-purpose mining machines with higher performance will be developed inevitably, and ordinary people cannot participate in mining. The CPoST mechanism is much more accessible because of slow iteration of hard disk manufacturers and low entry barrier. In traditional businesses, the vendor is normally not a competitor to users. But in the PoW systems, the ASIC manufacturer is the biggest miner. It can be easily understood that the miner's competitor is the miner's vendor, since device suppliers take most of the profit by providing mining machines, miner is radically the risk-free arbitrage of ASIC manufacturers.
\end{flushleft}
\subsection{Power Resources Monopoly}
\begin{flushleft}
    The power resources monopoly leads to no PoW ecosystem expansion, as the cost of mining exceeds the return. For those CPoST miners, the hard disks have much lower power requirement, thus the return of mining is higher. The linear hedge ratio of civil computer hardware can also be taken into consideration to ensure that miners can hedge the price fluctuation risk in the secondary market under the condition of relative safety and cost protection.
\end{flushleft}
\section{DePINC Architecture and Consensus Mechanism}
\subsection{Bitcoin and Chia}
\begin{flushleft}
    DePINC is derived from Bitcoin and the consensus from Chia's PoST. Bitcoin started in Jan 2009, the stability of wallet and blockchain is widely accepted after 10 years of iterations, it is safe and reliable to implement the PoS consensus on the Bitcoin QT wallet. DePINC also inherits Bitcoin's excellent P2P network architecture and UTXO system, which is mature and stable. The wallet client could implement any latest developments from the Bitcoin community: lightning network, script upgrades, and much more. The interface standard is kept same as that of Bitcoin, allowing users to integrate easily.
\end{flushleft}
\begin{flushleft}
    Chia started in August 2017. Combining the advantages of Bitcoin and Chia, DePINC has currently become the most reliable public chain with PoST consensus algorithm. Since its launch on August 3rd 2018, DePINC has grown steadily in computing power, withstood numerous tests, attacks, and cracks, and so far no major loopholes have emerged. By adopting the mature PoST mechanism, a stable and reliable consensus mechanism is introduced to build community confidence in the DePINC public chain. Since being compatible with Chia Plot files, miners can get both DePINC and Chia benefits, with only an additional operation.
\end{flushleft}
\subsection{CPoST Model}
\begin{flushleft}
    The CPoST ecosystem model includes mining pool, miner, crypto currency holder, wallet, exchanges and hardware vendor. The positive inner cycle and entrance of outside resources would bring expansion and development to this ecosystem, the rising price of DePINC would attract more miners; more miners coming to the system will lead to further price increase.
\end{flushleft}
\begin{flushleft}
    The cost of PoW is influenced by four factors: cost of dishonesty, cost of mining, level of difficulty and cost of mining devices. In the end, the PoW would become another low gross margin industry, the former windfall profits was because of insufficient scale, fluctuation of secondary market and limited device vendors. When it comes to PoST, due to the relatively low power consumption by hardware, miners can obtain other coins in the future symbiotic ecosystem of PoST almost free of charge without any risk.
\end{flushleft}
\begin{flushleft}
    The CPoST system, could give miners the choice to have most of the profits, incur cost for them to be the holder of other PoST coins, and avoid any malicious act. At the same time, the CPoST system attaches great importance to the release of distributional right and packaging right without barrier, which brings equity to the system. DePINC network architecture and the participants:
\end{flushleft}
\begin{flushleft}
    \centering\textbf{DePINC network architecture}
\end{flushleft}
\input{graph-depc_network_arch}
\subsection{Miners Mining Procedure}
\subsubsection{Plot}
\begin{flushleft}
    Miner plots file at local hard disk, and uses hash value to fill the disk. The larger the storage space, the more hash value could be filled, and higher block generation rate. Hash algorithm uses Shabal256, which is anti-ASIC.
\end{flushleft}
\subsubsection{Transaction}
\begin{flushleft}
    Wallet makes up the P2P network(inherited from BTC): Transactions happen between wallets.
\end{flushleft}
\subsubsection{Forging}
\begin{flushleft}
    Miner use wallet to listen to the P2P network, once a block is received, the packaging process of the next block starts. Wallet composes a block, sends the hash value of the block to miner, then miner finds the matching nonce. Once wallet receives nonce, it turns the nonce to deadline, wait for the time to end and then broadcast the block.
\end{flushleft}
\subsubsection{Verify}
\begin{flushleft}
    Receives the block, verifies it.
\end{flushleft}
\subsection{Principle and algorithms}
\subsubsection{Algorithms and acronyms}
\begin{flushleft}
    Decenterialized consensus algorithms require to consume scarce resources, such as computing power, staked money and storage space. DePINC is continuing to use storage space to secure the network. Timelord has been added to provide a reliable cryptographic. By adding timelord to increase the security of entired network.
\end{flushleft}
\begin{flushleft}
    DePINC cryptocurrency system combines Proof of Space (PoS) and Time (PoT). All required proofs are submitted to block-chain and can be easily verified. Miners need to find the correct answer by searching those random-looking data to win the lottery. No funds, special hardware, registration or permission is required to join except a hard driver and internet connection.
\end{flushleft}
\subsubsection{Proof of Space}
\begin{flushleft}
    A Proof of Space protocol is one in which:
\end{flushleft}
\begin{itemize}
    \item A Verifier can send a challenge to a Prover.
    \item The Prover can demonstrate to the Verifier that the Prover is reserving a specific amount of storage space at that precise time.
\end{itemize}
\begin{flushleft}
    The Proof of Space protocol has three components: plotting, proving/farming, and verifying. For more info, see Chia's Details of the chiapos specification, and reference implementation by visiting https://chia.net.
\end{flushleft}
\input{graph-pos_prover_verifier}
\subsection{Plotting}
\begin{flushleft}
    Plotting is the process by which a Prover, who we refer to as a miner, initializes a certain amount of space. To become a miner, one must have at least 101.4 GiB available to reserve on their computer (the minimum spec is a Raspberry Pi 4). There is no upper limit to the size of a Chia farm. Several farmers have multi-PiB farms.
\end{flushleft}
\begin{flushleft}
    As of Chia 1.2.7, a k32 plot can be created in around five minutes with a high-end machine with 400 GiB of RAM, or six hours with a normal commodity machine, or 12 hours with a slow machine using one CPU core and a few GB of RAM. Opportunities still remain for huge speedups. Furthermore, each plot only needs to be created once; a miner can farm with the same plots for many years.
\end{flushleft}
\begin{flushleft}
    Plot sizes are determined by a k parameter, where $space = 780 * k * 2^{k - 10}$, with a minimum k of 32 (101.4 GiB). The Proof of Space construction is based on Beyond Hellman, but it is nested six times (thereby creating seven tables), and it contains other heuristics to make it practical.
\end{flushleft}
\begin{flushleft}
    Each of the seven tables in a plot is filled with random-looking data that cannot be compressed. Each table has $2^k$ entries. Each entry in table i contains two pointers to table i-1 (the previous table). Finally, each table-1 entry contains a pair of integers between 0 and $2^k$, called "x-values." A Proof of Space is a collection of 64 x-values that have a certain mathematical relationship. The actual on-disk structure and the algorithm required to generate it are quite complicated, but this is the general idea.
\end{flushleft}
\begin{flushleft}
    Once the Prover has initialized 101.4 GiB, they are ready to receive a challenge and create a proof. One attractive property of this scheme is that it is non-interactive: no registration or online connection is required to create a plot using the original plot format. Nothing hits the blockchain until a reward is won, similar to PoW. For pool portable plots, a miner only needs a few mojos to create a plot NFT before plotting and then everything has the same characteristics from there.
\end{flushleft}
\subsection{Farming}
\begin{flushleft}
    Farming is the process by which a miner receives a sequence of 256-bit challenges to prove that they have legitimately put aside a defined amount of storage. In response to each challenge, the miner checks their plots, generates a proof, and submits any winning proofs to the network for verification.
\end{flushleft}
\begin{flushleft}
    For each eligible plot (explained later), a miner uses the following procedure to generate a full Proof of Space. Keep in mind that a plot consists of 7 tables (T1-T7) of approximately the same size, as well as 3 checkpoint tables (C1-C3), which are much smaller:
\end{flushleft}
\begin{enumerate}
    \item The miner receives a challenge from the block-chain
    \item For each eligible plot, extract a k-sized value from the challenge, where k denotes the size of the plot (k32, k33, etc)
    \item Look in the C2 table for a location at which to start scanning the C1 table
    \item Scan the C1 table for the location at which to start scanning the C3 table
    \item Read either one or two C3 parks. The number of parks to read depends on the index and value calculated from the C1 table. This requires an average of 5000 reads (the maximum is 10 000). These are sequential reads of 4 bytes (for an average total of 20 KiB)
    \item Grab all the f7 entries matching the challenge value (which can be 0 or more), along with the index in the table at which they were found
    \item For each matching f7 value, read T7 at the same index where the f7 value was found in its own table, and grab that entry, which is an index into T6
    \item The T6 index contains one line point with two back pointers to T5, four to T4, eight to T3, sixteen to T2 and thirty-two to T1. Each back pointer requires 1 read, so a total of 64 disk reads (1 index from T7, 63 back pointers) are performed to fetch the whole tree of 64 x-values.
\end{enumerate}
\begin{flushleft}
    Since most proofs generated by this process are not good enough to be submitted to the network for verification, we can optimize this process by only checking one branch of the tree. This branch will return two of the 64 x-values. The position of the x-values will always be consecutive and will depend on current challenge (eg x0 and x1... or x34 and x35). We hash these x-values to produce a random 256-bit ``quality string.'' This is combined with the difficulty and the plot size to generate the required\_iterations. If the required\_iterations is less than every required\_iterations those are found from local storage or internet, the proof can be included in the blockchain. At this point, we look up the whole Proof of Space.
\end{flushleft}
\begin{flushleft}
    By only looking up one branch to determine the quality string, we can rule out most proofs. This optimization requires only around 7-9 disk seeks and reads, or about 70-90 ms on a slow hard drive.
\end{flushleft}
\begin{flushleft}
    \textbf{INFO}\\[5pt]
    \textit{``Throughout this website, we'll make a simple assumption that a single disk seek requires 10ms. In reality, this is typically 5-10ms, so we're using a conservative estimate.\\[5pt]
    The 10ms estimate also takes into account the time required to transfer data after the seek. While storage industry specs typically assume that large files are being transferred, this does not hold true for DePINC farming, where proof lookups only require a tiny amount of data to be transferred. Therefore, it's safe to assume the transfer is almost instant.''}
\end{flushleft}
\begin{flushleft}
    DePINC also uses a further optimization to disqualify a certain proportion of plots from eligibility for each challenge. This is referred to as the plot filter. The current requirement is that the hash of the plot ID, challenge starts with 9 zeros. This excludes 511 out of every 512 plots. The filter hurts everyone equally (except for replotting attackers), and is therefore fair.
\end{flushleft}
\begin{flushleft}
    The plot filter effectively reduces the amount of resources required for farming by 512x -- each plot only requires a few disk reads every few minutes. A miner with 1 PiB of storage (10,000 plots) will only have an average of 20 plots that pass the filter for each challenge. Even if these plots all are stored on slow HDDs, and connected to a single Raspberry Pi, the average time required to respond to each challenge will be less than two seconds. This is well within the limits to avoid missing out on any challenges.
\end{flushleft}
\begin{flushleft}
    Each plot file has its own unique private key called a plot key. The plot ID is generated by hashing the plot public key, the miner public key, and either the pool public key (for OG plots) or the pool contract puzzle hash (for pooled plots). The requirements for signing a Proof of Space depend on the type of plots being used. See the Plot Public Keys page for details on the keys used for plot construction.
\end{flushleft}
\begin{flushleft}
    In practice, the plot key is a 2/2 BLS aggregate public key between a local key stored in the plot and a key stored by the miner software. For security and efficiency, a miner may run on one server using this key and signature scheme. This server can then be connected to one or more harvester machines that store the actual plots. Farming requires the miner key and the local key, but it does not require the pool key, since the pool's signature can be cached and reused for many blocks.
\end{flushleft}
\subsection{Verifying}
\begin{flushleft}
    After the miner has successfully created a Proof of Space, the proof can be verified by performing a few hashes and making comparisons between the x-values in the proof. Recall that the proof is a list of 64 x-values, where each x-value is k bits long. For a k32 this is 256 bytes (2048 bits), and is therefore very compact. Verification is very fast, but not quite fast enough to be verified in Solidity on Ethereum (something that would enable trustless transfers between chains), since this verification requires blake3 and chacha8 operations.
\end{flushleft}
\section{Proof of time (VDFs)}
\begin{flushleft}
    A Verifiable Delay Function, also referred to as a Proof of Time or VDF, is a proof that a sequential function was executed a certain number of times.
\end{flushleft}
\begin{flushleft}
    \textbf{Verifiable}: This means that after performing the computation (which takes time), the Prover can create a very small proof in a very short time, and the Verifier can verify this proof without having to redo the whole computation.
\end{flushleft}
\begin{flushleft}
    \textbf{Delay}: This means that the Prover actually spent a real amount of time (although we don't know exactly how much) to compute the function.
\end{flushleft}
\begin{flushleft}
    \textbf{Function}: This means it's deterministic: computing a VDF on an input x always yields the same result y.
\end{flushleft}
\begin{flushleft}
    The key word here is "sequential", like hashing a number many times: hash(hash(hash(a))), etc. This means the prover cannot just add more machines to make the function execute faster. Therefore we can assume that computing a VDF requires real (wall-clock) time. The construction that we use is repeated squaring. The Prover must square a challenge x T times. This requires time ϴ(T). The Prover also must create a proof that this was performed properly.
\end{flushleft}
\begin{flushleft}
    Although the following details are not very important for understanding the consensus algorithm, the choice of what VDF to use is relevant, because if an attacker succeeds in obtaining a much faster machine, some attacks become possible.
\end{flushleft}
\input{graph-pot_prover_verifier}
\begin{flushleft}
    The VDF used by DePINC is repeated squaring in a class group of unknown order. There are two main ways to generate a large group that has an unknown order:
\end{flushleft}
\begin{enumerate}
    \item Use an RSA modulus, and use the integers mod N as a group. The order of the group is unknown if you can generate your modulus with many participating parties using an MPC ceremony.
    \item An easier approach is to use classgroups with a large prime discriminant, which are groups of unknown order. This does not require any complex or trusted setup, so we chose this option for DePINC.
\end{enumerate}
\begin{flushleft}
    To create one of these groups, one just needs a large, random, prime number. The drawbacks are that classgroup code is less tested in real life, and optimizations are less well-known than in RSA groups. We use the same initial element for the squaring (a=2, b=1 classgroup element), and instead use the challenge to generate a new random prime number for each VDF, which is used as the discriminant. The discriminant has a size of 1024 bits, which means the proof sizes are around 1024 bits. We use the Wesolowski scheme split into n (1<=n<=64) phases so that creating the proofs is very fast. Since the n-wesolowski proofs can be large, we replace them with 1-wesolowski proofs as soon as they are available. These are smaller, but require more time to make. The proofs themselves are not committed to on-chain, so they are replaceable.
\end{flushleft}
\subsection{Infusion}
\begin{flushleft}
    As a recap, VDFs take in an input, called a challenge, and produce an output, together with a proof that certifies that the function was evaluated correctly.
\end{flushleft}
\begin{flushleft}
    A value, in this context, can be thought of as a block with a Proof of Space. The value is combined with an output of a VDF, to generate a new value, which is used as the input/challenge for the next VDF. This is known as an infusion of a value into a VDF.
\end{flushleft}
\begin{flushleft}
    Therefore, we are chaining VDFs, but committing to a new value in between. This is used so that we have a linear progression of blocks, alternating proofs of space with proofs of time.
\end{flushleft}
\subsection{Challenge}
\begin{flushleft}
    The DePINC consensus algorithm relies on timelords running VDFs for each block, which are adjusted periodically (and automatically) to take around 3 minutes. During every block, challenges are generated according previous block, and a sort of mini lottery starts, where farmers check their plots for proofs of space. When farmers find a Proof of Space that qualifies, they broadcast it to the network after the VDF calculation with required\_iterations is finished.
\end{flushleft}
\begin{flushleft}
    A challenge is always a 256-bit hash. It is released when a new block has been added to blockchain. The challenge services both PoS and VDF.
\end{flushleft}
\subsection{Quality and iterations}
\subsubsection{Quality}
\begin{flushleft}
    The number of quality is used to represent the quality of a proof of space which is found from plots. According the quality, an iterations will be calculated and it controls how many times should be passed before miner can post the block with the PoS.
\end{flushleft}
\begin{equation}
    Quality = \frac{QualityString * Plot_{size}}{2^{256}}
\end{equation}
\subsubsection{Iterations}
\begin{flushleft}
    ``required\_iterations'' is the number that VDF should run with. To verify a VDF proof not just only verify the proof itself, we also need to ensure the number of iterations is exceeded the requirement from PoS. Smaller of the number means the better quality it is. The block contains with the proof with better quality will be released earlier than others.
\end{flushleft}
\begin{equation}
    Iters = Difficulty * Difficulty_{factor} * \frac{QualityString}{2^{256} * Plot_{size}}
\end{equation}
\begin{flushleft}
    QualityString is mixed from current challenge and proof. $\frac{QualityString}{2^{256}}$ is a number between 0 and 1, multiply the number with the size of plot file will get the quality of the proof. The quality is used to multiply with difficulty in order to get the number of iterations. $Difficulty_{factor}$ is a constant number to fix the iterations to a reasonable range.
\end{flushleft}
\subsection{Blocks}
\begin{flushleft}
    DePINC releases block every 3 minutes. The basic information are assembled into the block such as transactions, Proof of Space, VDF proofs and difficulty.
\end{flushleft}
\subsubsection{Block generation steps}
\begin{enumerate}
    \item Find Proof of Space from plots
    \item Calculate ``required\_iterations'' according the proof
    \item Wait and retrieve VDF proof from timelord
    \item Create new block and pack with all proofs and TXs etc
\end{enumerate}
\input{graph-block_generation}
\subsubsection{Void block}
\begin{flushleft}
    There is a very rare situation, the proof of space cannot be found from the entired network. The consensus will add an empty duration without a PoS called void block. After the proof of VDF is calculated, miner will be asked to mix a new challenge to find a Proof of Space. The void block will be included into the new block.
\end{flushleft}
\subsection{Difficulty}
\begin{flushleft}
    Difficulty is the value represents how good is the block. According to current netspace and VDF speed, difficulty will be adjusted on every block. The time to release a new block will be around 3 minutes.
\end{flushleft}
\subsubsection{Difficulty adjustment}
\begin{flushleft}
    The equation of the difficulty adjustment is trying to calculate the new weight of next block. Adjusting difficulty also affects the number of iterations, this is also the way how the new block releases after 3 minutes.
\end{flushleft}
\begin{equation}
    NewDifficulty = \frac{Weight_{current} - Weight_{previous}}{Time_{CurrentBlock} * Time_{PreferBlock}}
\end{equation}
\subsection{Network space}
\begin{flushleft}
    Network space is the value that represents total amount of space those are allocated to generate current blockchain. Network space can be calculated from the difficulty of the last block on the chain.
\end{flushleft}
\begin{equation}
    NetSpace = \frac{Difficulty_{current}}{Iters_{current}}*{DifficultyConstantFactor}*2^{FilterBits}
\end{equation}
\begin{flushleft}
    \textit{Please note: the calculated size is not exactly the value of the network space. It is just the rule to mesure the network.}
\end{flushleft}
\section{Block validation}
\subsection{Block format}
\subsubsection{Proof of Space}
\begin{flushleft}
    To verify a PoS, the ``PlotId'' from plot is required, and we also need the public-key of the farmer (aka ``farmer-pk'') to verify the signature later. The plot also needs to be verified to ensure it is owned by the farmer. Record all related fields is the best way to accomplish it.
\end{flushleft}
\begin{tabular}{ |p{3cm}|p{3cm}|p{6cm}| }
    \hline
    \rowcolor{lightgray}\textbf{Name} & \textbf{Data type} & \textbf{Description} \\[5pt]
    \hline
    Pool pk or Hash & 48 or 32 bytes & According Chia's consensus.\\[5pt]
    \rowcolor{lightgray!30} Pk type & 1-byte & The type of the public key (OGPlot or PooledPlots) \\[5pt]
    Local pk & 48-byte & Local public-key identify the plot provide the proof \\[5pt]
    \rowcolor{lightgray!30} Farmer pk & 48-byte & Identify the farmer \\[5pt]
    K & 1-byte & The size of the plot file \\[5pt]
    \rowcolor{lightgray!30} Proof & multi-bytes & The proof of space \\[5pt]
    \hline
\end{tabular}
\subsubsection{VDF proofs}
\begin{flushleft}
    Verify VDF proof is more easier. Provide ``Proof'', ``Y'', ``witness type'' and ``Iterations'' to verify function will get the result. And the verifier also need to ensure the number of iterations is enough to satisify the consensus.
\end{flushleft}
\begin{tabular}{ |p{3cm}|p{3cm}|p{6cm}| }
    \hline
    \rowcolor{lightgray}\textbf{Name} & \textbf{Data type} & \textbf{Description} \\[5pt]
    \hline
    Y & multi-bytes & large prime discriminant \\[5pt]
    \rowcolor{lightgray!30} Proof & multi-bytes & The proof of time \\[5pt]
    Witness type & 1-byte & The type of witness \\[5pt]
    \rowcolor{lightgray!30} Iterations & 64-bit & The number of iterations \\[5pt]
    \hline
\end{tabular}
\subsubsection{Farmer signature}
\begin{flushleft}
    Farmer signature is used to ensure the owner of the proof of space. The signature can be verified by ``farmer-pk''. The number of bytes of the signature is 96.
\end{flushleft}
\subsubsection{Quality}
\begin{flushleft}
    A 64-bit number represents the quality of proof of space. The way to verify the quality is use the equation we mention before.
\end{flushleft}
\subsubsection{Difficulty}
\begin{flushleft}
    Difficulty is a 64-bit constant number represents the network space. The difficulty can be calculated from previous block.
\end{flushleft}
\subsection{Verify block}
\begin{flushleft}
    The following steps list all required checks to ensure the validity of a block.
\end{flushleft}
\begin{enumerate}
    \item Check previous block - To ensure the challenge is correct and generated from the previous block
    \item Check duration of VDF - The duration between blocks must has a reasonable value
    \item Check number of iterations of VDF - The number of iterations must satisify the requirement
    \item Check void blocks - Ensure the void blocks are valid and the challenge is mixed correctly
    \item Check difficulty - The difficulty can be calculated from the previous block
    \item Check the quality of the proof
    \item Verify the proof of space
    \item Verify the proof of VDF
    \item Verify farmer's signature
    \item Check distributed amount by coinbase
    \item Check validity of pledges TXs
\end{enumerate}
\section{Economic model}
\begin{flushleft}
    Economic has been improved after the Chia's consensus is updated. The total supply amount is increased and add ``lock period'' for each pledge.
\end{flushleft}
\subsection{Total supply is increased}
\begin{flushleft}
    The amount of total supply is increased to 84,000,000.
\end{flushleft}
\begin{itemize}
    \item The total amount to be supplied in each block will be doubled after consensus updated
    \item The foundation will receive an additional amount of the total amount multiply with 2 already supplied on the current chain at one time
    \item The Foundation will no longer receive funds in future blocks.
\end{itemize}
\subsection{Pledge improvement}
\begin{flushleft}
    The miner needs to pledge a certain amount of DePINC to the chain, and the pledged amount is related to its pledge time. When the pledge time is less than three years, the pledge amount will be discounted. During the pledge time, these DePINC will not be able to be withdrawn.
\end{flushleft}
\subsubsection{Netspace}
\begin{flushleft}
    Assuming that the netspace of a miner is ``p'', the current netspace of the entire network is ``t'', and the current distributed amount is ``m'', then the current miner wants to achieve the condition of full pledge, and the amount of currency ``a'' needs to pledge is:
\end{flushleft}
\begin{equation}
    a = \frac{p}{t} * m
\end{equation}
\subsubsection{Lock period}
\begin{flushleft}
    Now the miner needs to select the type of period for every new pledge. It determines the ratio value of the total amount that the pledge amount actual is.
\end{flushleft}
\begin{tabular}{ |p{5cm}|p{3cm}|p{3.5cm}| }
    \hline
    \rowcolor{lightgray}\textbf{Period} & \textbf{Ratio} & \textbf{$1_{DePINC}$ required} \\[5pt]
    \hline
    Current Deposit (1 week) & $8\%$ & $12.5_{DePINC}$ \\[5pt]
    \rowcolor{lightgray!30} 1 year & $20\%$ & $5_{DePINC}$ \\[5pt]
    2 years & $50\%$ & $2_{DePINC}$ \\[5pt]
    \rowcolor{lightgray!30} 3 years & $100\%$ & $1_{DePINC}$ \\[5pt]
    \hline
\end{tabular}
\subsubsection{Burn}
\begin{flushleft}
    When the pledge period has not yet expired, it is allowed to lose part of the pledged currency to withdraw the pledged amount, but part of the currency will be destroyed according to the pledged time. Assuming that ``p'' blocks have been pledged, a total of ``f'' blocks need to be pledged, and the amount of pledged currency is ``a''. Then, withdrawing the pledge with the amount of currency a in block ``p'' will return the amount ``w'' back, and $a-w$ is the amount of currency destroyed after this withdrawal.
\end{flushleft}
\begin{equation}
    w = \frac{p}{f} * a
\end{equation}
\subsubsection{Burning mechanism}
\begin{flushleft}
    DePINC will use a special (20-byte) accountID, it is filled by 0x23 of each byte. No one owns the private-key, and the consensus also prevent that no one will be able to make a new transaction that transfer DePINC from this account even the foundation. All burned DePINC will be sent to this account, and the amount required of mining will be re-calculated on next block.
\end{flushleft}
\subsubsection{Total amount update frequency}
\begin{flushleft}
    The number of total distributed amount is calculated once a month (total 20*24*30 heights), and the data will remain unchanged for one month until the next calculation. This means that the amount of currency that all miners need to pledge also changes every month. This mechanism simplifies the calculation of pledges, and miners no longer need to calculate the total distributed amount frequently.
\end{flushleft}
\subsubsection{Insufficient pledge amount}
\begin{flushleft}
    When the amount of pledge is not enough for a miner to claim all rewards from a new block, miners will only get $15\%$ of the block rewards, the rest $85\%$ will be locked in the chain until the miner who has enough amount of pledge to claim all rewards.
\end{flushleft}
\subsubsection{For example}
\begin{flushleft}
    We assume that there is a miner initialized $15_{PB}$ storage to do the mining. Assume that the current netspace of the entire network is $200_{PB}$ and the total supplied amount is $10,000_{DePINC}$. Thus, according to the formula, the total amount that needs to be pledged per PB is $\frac{10,000}{200}=50_{DePINC}$. The miner in order to obtain $100\%$ mining rewards, a total of $50_{DePINC} * 15_{PB} = 750_{DePINC}$ needs to be pledged to the chain.
\end{flushleft}
\textbf{$100\%$ rewards}
\begin{flushleft}
    The miner can choose one of the following pledge plans to get $100\%$ rewards:
    \begin{enumerate}
        \item Deposit $9375_{DePINC}$ to the chain with period ``Current Deposit''
        \item Deposit $3750_{DePINC}$ to the chain with period ``1 year''
        \item Deposit $1500_{DePINC}$ to the chain with period ``2 years''
        \item Deposit $750_{DePINC}$ to the chain with period ``3 years''
    \end{enumerate}
\end{flushleft}
\textbf{$15\%$ rewards}
\begin{flushleft}
    Miner will receive $15\%$ rewards when the amount of pledge is less than $750_{DePINC}$. The pledge amount also increases when the total supply is increased or the miner initialized more space to do the mining.
\end{flushleft}
\subsubsection{Withdraw an unexpired pledge}
\begin{flushleft}
    The withdrawal amount will be calculated according to the percentage of the pledged period when the pledge is withdrawn unexpired. The remaining amount will be directly burned on the blockchain. For example: there is a  pledge with total amount of $300_{DePINC}$, the period of this pledge is 3 years. If we want to withdraw it after only 10 months. There are only a total of $83.33_{DePINC}$ will be withdrawn, the remaining $216.67_{DePINC}$ Will be burned.
\end{flushleft}
\begin{flushleft}
    Formula to calculate the amount
\end{flushleft}
\begin{equation}
    Withdrawal_{DePINC} = Total_{DePINC} * \frac{Time_{Elapsed}}{Time_{Agreed}}
\end{equation}
\subsection{Foundation addresses}
\begin{flushleft}
    There is at least one foundation address to be able to use for generating new blocks without binding them with a valid farmer public-key. This is a mechanism to ensure that the network will keep generating new blocks and this allows new miner to create binding/pointing transaction to add miners to the network. Consensus ensures that the pledge amount of foundation addresses will not be able to calculate with full mortgage, even someone deposit pledge to the foundation addresses.
\end{flushleft}
